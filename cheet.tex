\documentclass[a4paper,12pt]{article}
\usepackage[utf8]{inputenc}
\usepackage[russian]{babel}
\usepackage{geometry}

\geometry{top=2cm}
\geometry{bottom=1cm}
\geometry{left=2cm}
\geometry{right=2cm}

\begin{document}

\section{Зачем?}
\begin{enumerate}
\item Для вычисления площади поверхности молекулы
Площадь поверхности контакта двух молекул позволяет оценить
их взаимодействие и, следовательно, стабильность комплекса
\item Для визуализации на поверхности электростатического потенциала,
гидрофобных областей и других характеристик
Помогает предсказывать области белка, взаимодействующие с
другими молекулами, проверять корректность моделей
\item Для выявления полостей, каналов в белке, карманов и т.п.
Следовательно, доступных для воды, ионов, лигандов
\item Для выявления остатков, экспонированных на поверхности белка
Если в одном белке область важна для взаимодействия с
другой молекулой, то для похожей области в другом белке
можно предсказать подобное же взаимодействие
\item Для поиска сходных областей поверхности
расчет энергии сольватации,  симуляция молекулярной динамики, докинг
\end{enumerate}

\section{SAS}
Поверхность, доступная для воды, использовалась, например, для того,
чтобы показать, какие аминокислотные остатки чаще экспонированы --
доступны для воды

С точки зрения геометрии, SAS аналогична vdWS, но с большими радиусами

\section{MS}
В пространстве MS лежит между vdWS и SAS

\section{Алгоритм Connolly}
\begin{enumerate}
\item Поверхность контакта
на поверхности каждой VdW сферы атома белка строится
равномерная сеть точек;
для каждой точки проверяется, что молекула воды,
касающаяся этой точки, не пересекается с белком;
если пересекается, то точка удаляется.
\item Дополнительная поверхность -- тороидальная
Каждая пара соседних атомов определяет тороидальную поверхность
между ними
На этой поверхности строится равномерная сеть точек
Далее – как для контактной поверхности
\item Дополнительная поверхность -- сферическая
Каждая тройка соседних атомов определяет сферическую
дополнительную поверхность -- ван-дер-ваальсову поверхность
молекулы воды, касающейся этих атомов
Если эта молекула воды не пересекается с белком, то на
подходящей части этой поверхности строится равномерная сеть точек
\end{enumerate}

\section{Contact}
вклад взаимодействия макромолекул (или частей макромолекул)
в энергию системы примерно пропорционален площади,
скрытой при взаимодействии, по сравнению с
невзаимодействующими молекулами (или частями молекул)

\end{document}

